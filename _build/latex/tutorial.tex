%% Generated by Sphinx.
\def\sphinxdocclass{report}
\documentclass[letterpaper,10pt,english]{sphinxmanual}
\ifdefined\pdfpxdimen
   \let\sphinxpxdimen\pdfpxdimen\else\newdimen\sphinxpxdimen
\fi \sphinxpxdimen=.75bp\relax

\usepackage[utf8]{inputenc}
\ifdefined\DeclareUnicodeCharacter
 \ifdefined\DeclareUnicodeCharacterAsOptional
  \DeclareUnicodeCharacter{"00A0}{\nobreakspace}
  \DeclareUnicodeCharacter{"2500}{\sphinxunichar{2500}}
  \DeclareUnicodeCharacter{"2502}{\sphinxunichar{2502}}
  \DeclareUnicodeCharacter{"2514}{\sphinxunichar{2514}}
  \DeclareUnicodeCharacter{"251C}{\sphinxunichar{251C}}
  \DeclareUnicodeCharacter{"2572}{\textbackslash}
 \else
  \DeclareUnicodeCharacter{00A0}{\nobreakspace}
  \DeclareUnicodeCharacter{2500}{\sphinxunichar{2500}}
  \DeclareUnicodeCharacter{2502}{\sphinxunichar{2502}}
  \DeclareUnicodeCharacter{2514}{\sphinxunichar{2514}}
  \DeclareUnicodeCharacter{251C}{\sphinxunichar{251C}}
  \DeclareUnicodeCharacter{2572}{\textbackslash}
 \fi
\fi
\usepackage{cmap}
\usepackage[T1]{fontenc}
\usepackage{amsmath,amssymb,amstext}
\usepackage{babel}
\usepackage{times}
\usepackage[Bjarne]{fncychap}
\usepackage[dontkeepoldnames]{sphinx}

\usepackage{geometry}

% Include hyperref last.
\usepackage{hyperref}
% Fix anchor placement for figures with captions.
\usepackage{hypcap}% it must be loaded after hyperref.
% Set up styles of URL: it should be placed after hyperref.
\urlstyle{same}
\addto\captionsenglish{\renewcommand{\contentsname}{Contents:}}

\addto\captionsenglish{\renewcommand{\figurename}{Fig.}}
\addto\captionsenglish{\renewcommand{\tablename}{Table}}
\addto\captionsenglish{\renewcommand{\literalblockname}{Listing}}

\addto\captionsenglish{\renewcommand{\literalblockcontinuedname}{continued from previous page}}
\addto\captionsenglish{\renewcommand{\literalblockcontinuesname}{continues on next page}}

\addto\extrasenglish{\def\pageautorefname{page}}

\setcounter{tocdepth}{1}



\title{tutorial Documentation}
\date{Mar 05, 2020}
\release{0.1}
\author{prettyboy}
\newcommand{\sphinxlogo}{\vbox{}}
\renewcommand{\releasename}{Release}
\makeindex

\begin{document}

\maketitle
\sphinxtableofcontents
\phantomsection\label{\detokenize{index::doc}}



\chapter{2020 学习计划}
\label{\detokenize{usage/installation:id1}}\label{\detokenize{usage/installation::doc}}
创建2020的学习计划


\section{3月}
\label{\detokenize{usage/installation:id2}}
\sphinxstylestrong{3月目标}
\begin{itemize}
\item {} 
把每天的事情都记录下来

\item {} 
有空写博客

\end{itemize}


\section{4月}
\label{\detokenize{usage/installation:id3}}
\sphinxstylestrong{4月目标}
\begin{itemize}
\item {} 
ROS编程

\item {} 
leetCode刷题

\end{itemize}


\chapter{0301 Sphinx 工具学习}
\label{\detokenize{usage/quickstart::doc}}\label{\detokenize{usage/quickstart:sphinx}}
Sphinx 是一个工具,它使得创建一个智能而美丽的文档变得简单。作者Georg Brandl,基于BSD许可证。
Sphinx 使用 reStructuredText 作为编写语言,也可以使用 Markdown + 拓展库的方式进行文档的编写。


\section{安装}
\label{\detokenize{usage/quickstart:id1}}
需要python3环境


\section{常用语法}
\label{\detokenize{usage/quickstart:id2}}
Subtitles are set with ‘-‘ and are required to have the same length
of the subtitle itself, just like titles.

Lists can be unnumbered like:
\begin{itemize}
\item {} 
Item Foo

\item {} 
Item Bar

\end{itemize}

Or automatically numbered:
\begin{enumerate}
\item {} 
Item 1

\item {} 
Item 2

\end{enumerate}

Words can have \sphinxstyleemphasis{emphasis in italics} or be \sphinxstylestrong{bold} and you can define
code samples with back quotes, like when you talk about a command: \sphinxcode{sudo}
gives you super user powers!

表格


\begin{savenotes}\sphinxattablestart
\centering
\sphinxcapstartof{table}
\sphinxcaption{:header:参数,类型,含义
:widths:2,2,5}\label{\detokenize{usage/quickstart:id5}}
\sphinxaftercaption
\begin{tabulary}{\linewidth}[t]{|T|T|T|}
\hline

test1
&
String
&
这里是测试的第一行
\\
\hline
test2
&
int
&
这里是测试的第二行
\\
\hline
\end{tabulary}
\par
\sphinxattableend\end{savenotes}

代码块

\fvset{hllines={, ,}}%
\begin{sphinxVerbatim}[commandchars=\\\{\}]
public void test()\PYGZob{}
    throws new Exception(\PYGZdq{}this is a test\PYGZdq{});
\PYGZcb{}
\end{sphinxVerbatim}

引用其他模块文件

点击跳转


\section{建立第一个工程}
\label{\detokenize{usage/quickstart:id3}}
直接命令行运行 \sphinxcode{sphinx-quickstart} ,按照向导进行建立工程

\noindent\sphinxincludegraphics{{tree}.png}

点击跳转

调用 \DUrole{xref,std,std-ref}{点击这里跳转}

编译方式:

\fvset{hllines={, ,}}%
\begin{sphinxVerbatim}[commandchars=\\\{\}]
\PYGZdl{}cd \PYGZti{}/Documents/sphinx\PYGZus{}tutorial
\PYGZdl{}make html
\end{sphinxVerbatim}


\section{配置github环境}
\label{\detokenize{usage/quickstart:github}}
\fvset{hllines={, ,}}%
\begin{sphinxVerbatim}[commandchars=\\\{\}]
git config \PYGZhy{}\PYGZhy{}global user.email \PYGZdq{}zuisixian@mail.com\PYGZdq{}
git config \PYGZhy{}\PYGZhy{}global user.name \PYGZdq{}zuisixian\PYGZdq{}
ssh\PYGZhy{}keygen \PYGZhy{}t rsa \PYGZhy{}C \PYGZdq{}zuisixian@mail.com\PYGZdq{}
\end{sphinxVerbatim}


\section{官网资料}
\label{\detokenize{usage/quickstart:id4}}\begin{enumerate}
\item {} 
\sphinxurl{https://www.sphinx-doc.org/en/master/devguide.html}

\item {} 
\sphinxurl{https://www.sphinx-doc.org/en/master/\#confval-language}

\item {} 
\sphinxurl{https://www.cnblogs.com/Terrypython/p/10203332.html}

\item {} 
\sphinxurl{https://www.sphinx-doc.org/en/master/man/sphinx-apidoc.html}

\item {} 
\sphinxurl{https://www.sphinx-doc.org/en/master/\#confval-language}

\item {} 
github: \sphinxurl{https://github.com/sphinx-doc/sphinx}

\item {} 
projects using Sphinx: \sphinxurl{https://www.sphinx-doc.org/en/master/examples.html}

\item {} 
getstarted: \sphinxurl{https://matplotlib.org/sampledoc/getting\_started.html\#installing-your-doc-directory}

\end{enumerate}


\section{reference}
\label{\detokenize{usage/quickstart:reference}}\begin{enumerate}
\item {} 
\sphinxurl{https://www.cnblogs.com/yqmcu/p/9837167.html}

\item {} 
\sphinxurl{https://blog.csdn.net/yeshennet/article/details/82595369}

\end{enumerate}



\renewcommand{\indexname}{Index}
\printindex
\end{document}